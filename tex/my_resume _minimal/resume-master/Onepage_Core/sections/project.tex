\cvsection{Projects}

\begin{cventries}

% \cventry
%    {Faculty Advisor: Prof. Aditya K. Jagannatham}
%    {\href{https://github.com/RohitRanjangit/5G-6G-development}{5G/6G Development \& ML}}
%    {}
%    {June 2020 - Present}
%    {
%      \begin{cvitems}
%        \item{Explored and Analysed various aspects of the existing wireless \textbf{2G, 3G, 4G} and \textbf{5G} systems \& it’s embedded technologies like \textbf{MIMO,OFDM} and \textbf{CDMA}}
%        \item{Designed a \textbf{Bayesian} \textbf{Decision} \textbf{model} using \textbf{Maximum} \textbf{likelihood} \textbf{estimation}, \textbf{Bayesian} \textbf{Classifiers} and \textbf{kernel} \textbf{based} \textbf{density} \textbf{estimation} to distinguish foreground and background of an \textbf{grayscale} image of a cheetah}
%        %\item{Implemented a novel image preprocessing algorithm based on Fusion Framework to formulate a robust underwater computer vision pipeline}
%        
%        \item{Acheived maximum \textbf{Accuracy} of \textbf{96.3\%} with \textbf{falseness: 0.153} for the above model in Image distinction}
%        % \item{Used signal processing operations such as Short Time Fourier Transform and Cross-Correlation to find time delay of arrival between signals}
%        % \item{Managed a multi-layered software stack for an autonomous underwater vehicle, Anahita developed on ROS and simulated using Gazebo}
%        \item{Used \textbf{MATLAB} machine learning toolbox, \textbf{SciPy}, \textbf{NumPy}, \textbf{Matplotlib} to implement above models and analyse their accuracy \& performance}
%        % \item{Created setups for disparity map generation using a pair of cameras and implemented a modified Fast-SLAM for underwater localization}
%      \end{cvitems}
%    }
  
   \cventry
    {Mentor: Prof. Manindra Agarwal, CSE}
    {\href{https://github.com/RohitRanjangit/CavesGame}{Caves Game}}
    {CS641 Course Project}
    {Jan 2020 - May 2020}
    {
      \begin{cvitems}
        \item {Explored and Analysed different existing classical and modern \textbf{Cryptographic methods} and their weaknesses}
        \item {Completed all \textbf{7} levels of game by designing \textbf{chosen plaintext attack} for weaker models of \textbf{AES},\textbf{DES} and \textbf{RSA} and extracted the keys used to encrypt data}
      \end{cvitems}
    }

  \cventry
    {Mentor: Aditya Gulati}
    {\href{https://github.com/life-iitk}{Life@IITK}}
    {Science \& technology Council}
    {May 2019- July 2019}
    {
      \begin{cvitems}
        \item {Collaborated with application developers team to create a \textbf{web application} which streamlines the various aspects of the day-to-day lives of campus students}
        \item {Worked with frontend team to design a \textbf{Map Page} using \textbf{ReactJS} showing ongoing events in IITK with pinned location on map according to building or place where events are going to happen}
        \item Also assisted backend team to establish data tables of users and events \& create relations between these
        \item {Used \textbf{Django-REST} to create a \textbf{Rest-API} which helps in serialization of events data}
      \end{cvitems}
    }

    \cventry
    {Mentor: Mrinaal Dogra}
    {\href{https://github.com/RohitRanjangit/P2P-video-conferencing-app}{P2P Video conferencing App}}
    {ACA,CSE dept.- IIT Kanpur}
    {Jan 2019- Mar 2019}
    {
      \begin{cvitems}
        \item Designed a basic \textbf{web application} which connects multiple registered users and allows them to communicate between each other via text messages, voice call \& video call
        \item Used a Javascript open framework \textbf{WebRTC} to establish real-time communication between peers and enabling them to talk seamlessly
        \item Used \textbf{Python-Flask} framework to handle all backend necessary processes and methods(e.g \textbf{User Registeration},\textbf{ Sign
        In}, \textbf{Sign Off..} )
      \end{cvitems}
    } 
	 \cventry
	{Mentor: Sarthak Singhal, Aniket Sanghi}
	{Algorithms in Depth}
	{S\&T council- IIT Kanpur}
	{May 2019- Jul 2019}
	{
		\begin{cvitems}
			\item Studied and analysed the flow \& structure of various classical algorithms such as \textbf{Graph Algorithms}, \textbf{Data Compression Algorithms} and \textbf{Pattern matching Algorithms}
			\item Explored and implemented KMP, Aho-Carosick, Huffman Coding, Disjoint Set Union, floyd warshall and Bellman-Ford Algorithms.
		\end{cvitems}
	} 

  % \smallcventry
  %   {Robotics Club}
  %   {Team Humanoid, IITK}
  %   {IIT Kanpur}
  %   {Dec 2017 - April 2018} 
  %   {Software Team Member}
  %   {
  %     \begin{cvitems} 
  %       \item {Worked on a Bipedal Prototype of the humanoid bot, capable of performing statically stable walking}
  %       \item {Implemented the MATLAB simulated \textbf{inverse kinematics walking algorithm} based on ZMP criteria on the actual robot using a Robot Operating System framework}
  %       \item {Developed a Web Graphical User Interface for monitoring current status and easier debugging of servos using ROS Web Bridge Server and JavaScript, with a CSS frontend}

  %     \end{cvitems}
  %   }


\end{cventries}
\vspace{-2mm}

%%% Local Variables:
%%% mode: latex
%%% TeX-master: "../cv.tex"
%%% TeX-engine: xelatex
%%% End:


% \smallcventry
  % {Self Project}
  % {\href{https://github.com/yashsriv/go-nachos}{go-nachos}}
  % {Operating Systems}
  % {Dec'2017}
  % {}
  % {A port of the educational OS, nachos, in golang}

  % \smallcventry
  % {Course Project, Compiler Design}
  % {\href{https://github.com/yashsriv/tango}{tango} \strong{(\emph{golang to x86 assembly})} }
  % {}
  % {Jan'2018-April'2018}
  % {\emph{\texttt{\href{https://github.com/yashsriv/tango}{github://yashsriv/tango}}}}
  % {
  %   \begin{cvitems}
  %   \item A compiler for go written in go in a team of 3. Compiles from golang
  %     to x86 assembly.
  %   \item Supports a subset of the go language including nested pointers, type
  %     checking, recursion, nested arrays, structs, methods and other common
  %     programming language features.
  %   \item Added a new for comprehension syntax as well to golang.
  %   \end{cvitems}
  % }

  % \smallcventry
  % {Course Project, Computer Architecture}
  % {\href{https://github.com/yashsriv/branch-predictor/blob/master/report/main.pdf}{Branch Predictor}}
  % {Best Predictor}
  % {April'2018}
  % {\emph{\texttt{\href{https://github.com/yashsriv/branch-predictor/blob/master/report/main.pdf}{github://yashsriv/branch-predictor}}}}
  % {
  %   \begin{cvitems}
  %   \item Designed a branch predictor for an intra-class branch prediction
  %     championship based on the CBP-1 framework in a team of 2.
  %   \item Created a modified GEHL predictor with an additional loop predictor.
  %   \item Was adjudged the \textbf{best predictor} amongst all submitted.
  %   \end{cvitems}
  % }

  % \smallcventry
  % {Course Project}
  % {\href{https://github.com/yashsriv/haskell-connect-4}{Connect 4 AI in haskell}}
  % {Functional Programming}
  % {Jan'2018-April'2018}
  % {\emph{\texttt{\href{https://github.com/yashsriv/haskell-connect-4}{github://yashsriv/haskell-connect-4}}}}
  % {
  %   \begin{cvitems}
  %   \item A GUI based connect 4 AI in haskell.
  %   \item Had support for various difficulties and the AI was abstracted out in
  %     order to be able to support any complete knowledge two player game.
  %   \end{cvitems}
  % }

  % \smallcventry
  % {24 Hour Hackathon}
  % {Code.Fun.Do}
  % {Microsoft India}
  % {Sept'2015}
  % {Best 5 ideas}
  % {
  %   \begin{cvitems}
  %   \item Developed an App to help connect teachers and learners based on their
  %     preference of subjects.
  %   \item Used cross-platform \textbf{Universal App Platform} for Windows 10
  %     and a server written in C\#.
  %   \item Was selected as one of the \textbf{best five ideas}.
  %   \end{cvitems}
  % }

  % \smallcventry
  % {Self Project}
  % {\href{https://github.com/yashsriv/go-nachos}{go-nachos}}
  % {Ported nachOS to golang}
  % {Dec'2017}
  % {\emph{\texttt{\href{https://github.com/yashsriv/go-nachos}{github://yashsriv/go-nachos}}}}
  % {}