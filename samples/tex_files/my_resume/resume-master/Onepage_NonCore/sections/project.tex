\cvsection{Projects}

\begin{cventries}

 \cventry
    {Faculty Advisor: Prof. Mangal Kothari}
    {Team AUV-IITK}
    {Software Team Member}
    {May 2018 - Present}
    {
      \begin{cvitems}
        \item{Created \textbf{multi-class dataset} of labeled underwater photos, trained model and setup real-time inference on Jetson TX2}
        \item{Designed a \textbf{hierarichal finite state machine} for robust autonomous behavior of the vehicle with failsafes}
        \item{Fused sensor readings from Doppler Velocity Log (DVL) and IMU using an \textbf{EKF} to estimate odometry}
        %\item{Implemented a novel image preprocessing algorithm based on Fusion Framework to formulate a robust underwater computer vision pipeline}
        
        \item{Developed and tested acoustic localization system capable of estimating the Direction of Arrival of ultrasonic underwater signals from pinger, using \textbf{STFT} and \textbf{Cross-Correlation}}
        % \item{Used signal processing operations such as Short Time Fourier Transform and Cross-Correlation to find time delay of arrival between signals}
        % \item{Managed a multi-layered software stack for an autonomous underwater vehicle, Anahita developed on ROS and simulated using Gazebo}
        % \item{Tuned and tested Cascaded PID Controller on the vehicle, enabling it to perform waypoint navigation and visual servoing}
        \item{Extensively used \textbf{Gazebo, a physics engine} to simulate vehicle model in a hydrodynamically realistic environment}
        % \item{Created setups for disparity map generation using a pair of cameras and implemented a modified Fast-SLAM for underwater localization}
      \end{cvitems}
    }
   \cventry
    {Mentor: Prof. Mangal Kothari}
    {Realtime Onboard Dense RGB-D Mapping on UAVs}
    {}
    {May 2019 - Present}
    {
      \begin{cvitems}
        \item {Studied and experimented various techniques related to 3D mapping of environment using monocular and stereo cameras on Jetson TX2 for onboard implementation}
        \item {Evaluated approaches for shortcomings and processing requirements while focussing on the scarce size, computation and energy resources on Unmanned Aerial Vehicles (UAVs)}
      \end{cvitems}
    }

  \cventry
    {Mentor: Prof. Puroshottam Kar}
    {Chat-IITK}
    {Advanced Track Project - ESC101}
    {2nd Semester}
    {
      \begin{cvitems}
        \item {Designed and developed a chat application on NodeJS, Express, and MongoDB, selected in \textbf{12} out of 400+ students}
        \item {Implemented real-time chat using Socket-IO with PassportJS for extensively implemented \textbf{authentication} and \textbf{cookie handling} for session management}
        \item {\textbf{Database management} implemented using MongoDB, and application deployed online on Heroku's server}
      \end{cvitems}
    }

  \cventry
    {Robotics Club, IIT Kanpur}
    {Team Humanoid, IITK}
    {Software Team Member}
    {Dec. 2017 - April 2018} 
    {
      \begin{cvitems} 
        \item {Worked on a Bipedal Prototype of the humanoid bot, capable of performing statically stable walking}
        \item {Implemented the MATLAB simulated \textbf{inverse kinematics walking algorithm} based on ZMP criteria on the actual robot using a Robot Operating System framework}
        \item {Developed a Web Graphical User Interface for monitoring current status and easier debugging of servos using ROS Web Bridge Server and JavaScript, with a CSS frontend}

      \end{cvitems}
    }


  % \smallcventry
  %   {Mentor: Prof. Shantanu Bhattacharya}
  %   {Mechanical Quadruped}
  %   {Instructor}
  %   {4th Semester}
  %   {Course Project -TA202}
  %   {
  %     \begin{cvitems}
  %       \item Designed and simulated a four-legged assembly that uses Jansen's linkage mechanism to walk using Solidworks
  %       \item Made a working model of the same under constraints of size and materials using manufacturing processes such as lathing, milling and drilling
  %     \end{cvitems}
  %   } 

\end{cventries}
\vspace{-2mm}

%%% Local Variables:
%%% mode: latex
%%% TeX-master: "../cv.tex"
%%% TeX-engine: xelatex
%%% End:


% \smallcventry
  % {Self Project}
  % {\href{https://github.com/yashsriv/go-nachos}{go-nachos}}
  % {Operating Systems}
  % {Dec'2017}
  % {}
  % {A port of the educational OS, nachos, in golang}

  % \smallcventry
  % {Course Project, Compiler Design}
  % {\href{https://github.com/yashsriv/tango}{tango} \strong{(\emph{golang to x86 assembly})} }
  % {}
  % {Jan'2018-April'2018}
  % {\emph{\texttt{\href{https://github.com/yashsriv/tango}{github://yashsriv/tango}}}}
  % {
  %   \begin{cvitems}
  %   \item A compiler for go written in go in a team of 3. Compiles from golang
  %     to x86 assembly.
  %   \item Supports a subset of the go language including nested pointers, type
  %     checking, recursion, nested arrays, structs, methods and other common
  %     programming language features.
  %   \item Added a new for comprehension syntax as well to golang.
  %   \end{cvitems}
  % }

  % \smallcventry
  % {Course Project, Computer Architecture}
  % {\href{https://github.com/yashsriv/branch-predictor/blob/master/report/main.pdf}{Branch Predictor}}
  % {Best Predictor}
  % {April'2018}
  % {\emph{\texttt{\href{https://github.com/yashsriv/branch-predictor/blob/master/report/main.pdf}{github://yashsriv/branch-predictor}}}}
  % {
  %   \begin{cvitems}
  %   \item Designed a branch predictor for an intra-class branch prediction
  %     championship based on the CBP-1 framework in a team of 2.
  %   \item Created a modified GEHL predictor with an additional loop predictor.
  %   \item Was adjudged the \textbf{best predictor} amongst all submitted.
  %   \end{cvitems}
  % }

  % \smallcventry
  % {Course Project}
  % {\href{https://github.com/yashsriv/haskell-connect-4}{Connect 4 AI in haskell}}
  % {Functional Programming}
  % {Jan'2018-April'2018}
  % {\emph{\texttt{\href{https://github.com/yashsriv/haskell-connect-4}{github://yashsriv/haskell-connect-4}}}}
  % {
  %   \begin{cvitems}
  %   \item A GUI based connect 4 AI in haskell.
  %   \item Had support for various difficulties and the AI was abstracted out in
  %     order to be able to support any complete knowledge two player game.
  %   \end{cvitems}
  % }

  % \smallcventry
  % {24 Hour Hackathon}
  % {Code.Fun.Do}
  % {Microsoft India}
  % {Sept'2015}
  % {Best 5 ideas}
  % {
  %   \begin{cvitems}
  %   \item Developed an App to help connect teachers and learners based on their
  %     preference of subjects.
  %   \item Used cross-platform \textbf{Universal App Platform} for Windows 10
  %     and a server written in C\#.
  %   \item Was selected as one of the \textbf{best five ideas}.
  %   \end{cvitems}
  % }

  % \smallcventry
  % {Self Project}
  % {\href{https://github.com/yashsriv/go-nachos}{go-nachos}}
  % {Ported nachOS to golang}
  % {Dec'2017}
  % {\emph{\texttt{\href{https://github.com/yashsriv/go-nachos}{github://yashsriv/go-nachos}}}}
  % {}