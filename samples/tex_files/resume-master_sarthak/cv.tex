%!TEX TS-program = xelatex
%!TEX encoding = UTF-8 Unicode
% Awesome CV LaTeX Template for CV/Resume
%
% This template has been downloaded from:
% https://github.com/posquit0/Awesome-CV
%
% Author:
% Claud D. Park <posquit0.bj@gmail.com>
% http://www.posquit0.com
%
% Template license:
% CC BY-SA 4.0 (https://creativecommons.org/licenses/by-sa/4.0/)
%


\documentclass[10pt, a4paper]{awesome-cv}
\geometry{left=1.4cm, top=.8cm, right=1.4cm, bottom=1.8cm, footskip=.5cm}
\fontdir[fonts/]

% Color for highlights
% Awesome Colors: awesome-emerald, awesome-skyblue, awesome-red, awesome-pink, awesome-orange
%                 awesome-nephritis, awesome-concrete, awesome-darknight
\colorlet{awesome}{awesome-darknight}
% Uncomment if you would like to specify your own color
% \definecolor{awesome}{HTML}{CA63A8}

% Colors for text
% Uncomment if you would like to specify your own color
% \definecolor{darktext}{HTML}{414141}
% \definecolor{text}{HTML}{333333}
% \definecolor{graytext}{HTML}{5D5D5D}
% \definecolor{lighttext}{HTML}{999999}

% Set false if you don't want to highlight section with awesome color
\setbool{acvSectionColorHighlight}{false}

% If you would like to change the social information separator from a pipe (|) to something else
\renewcommand{\acvHeaderSocialSep}{\quad\textbar\quad}


% Available options: circle|rectangle,edge/noedge,left/right
% \photo{./profile.png}
\name{Yash}{Srivastav}
\position{Senior Undergraduate{\enskip\cdotp\enskip}Computer Science and Engineering}
\address{Indian Institute of Technology, Kanpur}
\mobile{(+91) 705-413-3662}
\email{yash111998@gmail.com}
\homepage{yashsriv.org}
\github{yashsriv}
\linkedin{yashsriv}
% \twitter{@therealyashsriv}
% \quote{``There is no fate but what we make."}

\newcommand{\smallcventry}[6]{\cventry{#1}{#2}{#3}{#4}{#6}}
\newcommand{\specialcvsection}[1]{\cvsection{#1}}




\begin{document}
\makecvheader
\makecvfooter
  {}
  {}
  {\thepage}

\cvsection{Education}

\begin{cventries}

  \cventry
    {Bachelor Of Technology in Mechanical Engineering and Minor in Industrial Management and Engineering}
    {Indian Institute of Technology Kanpur}
    {Kanpur, India}
    {July 2017 - Present}
    {
      \begin{cvitems}
        \item {Cumulative Point Index: \textbf{9.2*/10.0}}
      \end{cvitems}
    }
  
  \cventry
    {Indian School Certificate Examination (Intermediate)}
    {City Montessori School}
    {Lucknow, India}
    {March 2017}
    {
      \begin{cvitems}
        \item {Overall Percentage: 95.5\%}
      \end{cvitems}
    }

  \cventry
    {Indian Certificate Of Secondary Education (High School)}
    {St. Francis' College}
    {Lucknow, India}
    {March 2015}
    {
      \begin{cvitems}
        \item {Overall Percentage: 95\%}
      \end{cvitems}
    }
    
\end{cventries}


\cvsection{Scholastic and Programming Achievements}
\begin{cvhonorsa}
{\Large
  

  \cvhonora
  {Received Academic Excellence Award}
  {2017 - 18}
  {}
  {2017}
  
  \cvhonora
  {All India Rank \textbf{169}}
  {Joint Entrance Exam Advanced among 200,000 candidates}
  {}
  {2017}

  \cvhonora
  {All India Rank \textbf{78}}
  {Joint Entrance Exam Mains}
  {}
  {2017}
}

\end{cvhonorsa}
\begin{cvhonors}
{\Large
  \cvhonor
  {Qualified for \textbf{ACM ICPC} Amritapuri regionals 2018, Secured \textbf{173$^{rd}$} rank in regionals}
  {}
  {2018}
  
  \cvhonor
  {Ranked \textbf{31$^{st}$} among the Indian teams in Google Hash Code 2019}
  {}
  {2017}

  \cvhonor
  {Rated 5 star on CodeChef with rating of 2026}
  {}
  {2017}
  
  \cvhonor
  {Expert on Codeforces with maximum rating 1682}
  {}
  {2017}

}
\end{cvhonors}

\cvsection{Work Experience}
\begin{cventries}
    
    \cventry
    {SDE Intern, DESFlow}
    {D.E. Shaw India Private Limited}
    {Hyderabad, India}
    {April 2020 - June 2020}
    {
      \begin{cvitems}
        \item {Built an attachment prediction system for the internal ticketing management system, DESFlow.}
        \item {Experimented with various keyword based and ML based algorithms and built a Machine Learning model using the combination of Na\"{i}ve Bayes and Graham's algorithm and achieved 90\% precision, 59\% recall and 90\% accuracy.}
        \item{Improved the Graham's algorithm and employed normalized keyword approach on top of the Machine Learning model to improve the recall to 67\% and precision to 91\%.}
        \item{Extended the solution to handle similar use cases like setting the due date and closing the request and also added the support for active learning for model to evolve over time.}
      \end{cvitems}
    }
        
    \cventry
    {Software Developer, Boost C++}
    {Google Summer of Code}
    {India}
    {May 2019 - August 2019}
    {
      \begin{cvitems}
        \item {Integrated Boost.Units with the existing base coordinate system and restructured all the classes to make them compatible with units to provide a robust astronomical coordinate system.}
        \item {Created the parser for binary table extension and ASCII table extension for FITS File system.}
        \item{Used Template Meta Programming in C++ to provide almost no run-time overhead and allow users to write scientifically infallible code by detecting all the errors at the compile time.}
      \end{cvitems}
    }

  \cventry
    {Full Stack Developer, Under Prof. Sandeep Shukla}
    {Summer of Code, IIT Kanpur}
    {IIT Kanpur, India}
    {May 2018 - July 2018}
    {
      \begin{cvitems}
        \item {Developed a dynamic and scalable web application using LAMP stack from scratch as an initiative to improve the medical system by keeping track of records of patients and their medical history.}
        \item {Implemented various functionalities like doctor could add a patient, update its records, refer it to another doctor, etc. and patient could search for doctors and the labs in his area.}
        \item {Developed a question-answer platform for the medical
system which had the functionality to filter the questions based on illness, search questions based on keywords, notifications for patients if their questions get answered, etc.}
        \item{Technologies and languages used: PHP, MYSQL, AJAX, HTML, Javascript, Microsoft Azure(for deployment).}
        % \item{Used Microsoft Azure to deploy the web application.}
        % \item{Conducted by Nutanix and UPSIDC.}
        % \item {Project approached by Professor for implementing in Health Center of institute.}
      \end{cvitems}
    }
   
  
\end{cventries}
\cvsection{Skills}
\ifdefined\ONEPAGE
\\
\textbf{Robotics}: ROS, OpenCV, Arduino, Gazebo, CUDA, Gym\\
\textbf{Design}: Solidworks 2018, AutoCAD, Inventor, LabVIEW\\
\textbf{Data Science}: Tensorflow, Keras, Scikit, MATLAB\\
\textbf{Programming Languages}: C++, Python, Scala, Javascript\\
\else
\fi
%%% Local Variables:
%%% mode: latex
%%% End:
\cvsection{Relevant Courses}

\ifdefined\ONEPAGE

% \textbf{CS:} Introduction to Programming(A$*$), Logic in Computer
% Science, Computer Organization, Data Structures and Algorithms, Computing
% Laboratories - 1(A$*$)
{\Large
\vspace{2.5mm}
\begin{tabular*}{\textwidth}{l l l}
  Introduction to Programming & Linear Algebra & Introduction to Logic\\   Probability for Computer Science &
Data Structures and Algorithms & Discrete Mathematics\\
Software Labs & Computer Organization & Theory of Computation \\ Algorithms-II &
  Introduction to Machine Learning & Operating Systems \\
  Computer Architecture($i$) & Database Management($i$) & Compiler Design($i$)
\end{tabular*}



% \textbf{Math}: Discrete Math, Probability and Statistics(A$*$)


{\footnotesize
    {{\large ~~$i$: Upcoming}}
}

\else
{\fontsize{11pt}{1em}\bodyfontlight\upshape\color{text}
  \begin{tabular*}{\textwidth}{l l l}
    Introduction to Programming(A$*$) & Discrete Mathematics  & Computer Organization \\
    Computer Architecture & Data Structures and Algorithms & Probability \& Statistics(A$*$) \\ 
    Computing Laboratories - 1(A$*$) & Computing Laboratories - 2(A$*$) & Compiler Design \\
    Functional Programming(A$*$) & Computer Systems Security & Computer Networks($i$)
  \end{tabular*}
}
{\fontsize{11pt}{1em}\footerfont\upshape\color{text}
  \begin{tabular*}{\textwidth}{ l l }
    \entrylocationstyle{A$*$: Grade for exceptional performance} & \entrylocationstyle{$i$: In progress}\\
  \end{tabular*}
}

\fi
}
%%% Local Variables:
%%% mode: latex
%%% TeX-engine: xetex
%%% TeX-master: "../cv"
%%% End:
\newpage
\cvsection{Projects}

\begin{cventries}

% \cventry
%    {Faculty Advisor: Prof. Aditya K. Jagannatham}
%    {\href{https://github.com/RohitRanjangit/5G-6G-development}{5G/6G Development \& ML}}
%    {}
%    {June 2020 - Present}
%    {
%      \begin{cvitems}
%        \item{Explored and Analysed various aspects of the existing wireless \textbf{2G, 3G, 4G} and \textbf{5G} systems \& it’s embedded technologies like \textbf{MIMO,OFDM} and \textbf{CDMA}}
%        \item{Designed a \textbf{Bayesian} \textbf{Decision} \textbf{model} using \textbf{Maximum} \textbf{likelihood} \textbf{estimation}, \textbf{Bayesian} \textbf{Classifiers} and \textbf{kernel} \textbf{based} \textbf{density} \textbf{estimation} to distinguish foreground and background of an \textbf{grayscale} image of a cheetah}
%        %\item{Implemented a novel image preprocessing algorithm based on Fusion Framework to formulate a robust underwater computer vision pipeline}
%        
%        \item{Acheived maximum \textbf{Accuracy} of \textbf{96.3\%} with \textbf{falseness: 0.153} for the above model in Image distinction}
%        % \item{Used signal processing operations such as Short Time Fourier Transform and Cross-Correlation to find time delay of arrival between signals}
%        % \item{Managed a multi-layered software stack for an autonomous underwater vehicle, Anahita developed on ROS and simulated using Gazebo}
%        \item{Used \textbf{MATLAB} machine learning toolbox, \textbf{SciPy}, \textbf{NumPy}, \textbf{Matplotlib} to implement above models and analyse their accuracy \& performance}
%        % \item{Created setups for disparity map generation using a pair of cameras and implemented a modified Fast-SLAM for underwater localization}
%      \end{cvitems}
%    }
  
   \cventry
    {Mentor: Prof. Manindra Agarwal, CSE}
    {\href{https://github.com/RohitRanjangit/CavesGame}{Caves Game}}
    {CS641 Course Project}
    {Jan 2020 - May 2020}
    {
      \begin{cvitems}
        \item {Explored and Analysed different existing classical and modern \textbf{Cryptographic methods} and their weaknesses}
        \item {Completed all \textbf{7} levels of game by designing \textbf{chosen plaintext attack} for weaker models of \textbf{AES},\textbf{DES} and \textbf{RSA} and extracted the keys used to encrypt data}
      \end{cvitems}
    }

  \cventry
    {Mentor: Aditya Gulati}
    {\href{https://github.com/life-iitk}{Life@IITK}}
    {Science \& technology Council}
    {May 2019- July 2019}
    {
      \begin{cvitems}
        \item {Collaborated with application developers team to create a \textbf{web application} which streamlines the various aspects of the day-to-day lives of campus students}
        \item {Worked with frontend team to design a \textbf{Map Page} using \textbf{ReactJS} showing ongoing events in IITK with pinned location on map according to building or place where events are going to happen}
        \item Also assisted backend team to establish data tables of users and events \& create relations between these
        \item {Used \textbf{Django-REST} to create a \textbf{Rest-API} which helps in serialization of events data}
      \end{cvitems}
    }

    \cventry
    {Mentor: Mrinaal Dogra}
    {\href{https://github.com/RohitRanjangit/P2P-video-conferencing-app}{P2P Video conferencing App}}
    {ACA,CSE dept.- IIT Kanpur}
    {Jan 2019- Mar 2019}
    {
      \begin{cvitems}
        \item Designed a basic \textbf{web application} which connects multiple registered users and allows them to communicate between each other via text messages, voice call \& video call
        \item Used a Javascript open framework \textbf{WebRTC} to establish real-time communication between peers and enabling them to talk seamlessly
        \item Used \textbf{Python-Flask} framework to handle all backend necessary processes and methods(e.g \textbf{User Registeration},\textbf{ Sign
        In}, \textbf{Sign Off..} )
      \end{cvitems}
    } 
	 \cventry
	{Mentor: Sarthak Singhal, Aniket Sanghi}
	{Algorithms in Depth}
	{S\&T council- IIT Kanpur}
	{May 2019- Jul 2019}
	{
		\begin{cvitems}
			\item Studied and analysed the flow \& structure of various classical algorithms such as \textbf{Graph Algorithms}, \textbf{Data Compression Algorithms} and \textbf{Pattern matching Algorithms}
			\item Explored and implemented KMP, Aho-Carosick, Huffman Coding, Disjoint Set Union, floyd warshall and Bellman-Ford Algorithms.
		\end{cvitems}
	} 

  % \smallcventry
  %   {Robotics Club}
  %   {Team Humanoid, IITK}
  %   {IIT Kanpur}
  %   {Dec 2017 - April 2018} 
  %   {Software Team Member}
  %   {
  %     \begin{cvitems} 
  %       \item {Worked on a Bipedal Prototype of the humanoid bot, capable of performing statically stable walking}
  %       \item {Implemented the MATLAB simulated \textbf{inverse kinematics walking algorithm} based on ZMP criteria on the actual robot using a Robot Operating System framework}
  %       \item {Developed a Web Graphical User Interface for monitoring current status and easier debugging of servos using ROS Web Bridge Server and JavaScript, with a CSS frontend}

  %     \end{cvitems}
  %   }


\end{cventries}
\vspace{-2mm}

%%% Local Variables:
%%% mode: latex
%%% TeX-master: "../cv.tex"
%%% TeX-engine: xelatex
%%% End:


% \smallcventry
  % {Self Project}
  % {\href{https://github.com/yashsriv/go-nachos}{go-nachos}}
  % {Operating Systems}
  % {Dec'2017}
  % {}
  % {A port of the educational OS, nachos, in golang}

  % \smallcventry
  % {Course Project, Compiler Design}
  % {\href{https://github.com/yashsriv/tango}{tango} \strong{(\emph{golang to x86 assembly})} }
  % {}
  % {Jan'2018-April'2018}
  % {\emph{\texttt{\href{https://github.com/yashsriv/tango}{github://yashsriv/tango}}}}
  % {
  %   \begin{cvitems}
  %   \item A compiler for go written in go in a team of 3. Compiles from golang
  %     to x86 assembly.
  %   \item Supports a subset of the go language including nested pointers, type
  %     checking, recursion, nested arrays, structs, methods and other common
  %     programming language features.
  %   \item Added a new for comprehension syntax as well to golang.
  %   \end{cvitems}
  % }

  % \smallcventry
  % {Course Project, Computer Architecture}
  % {\href{https://github.com/yashsriv/branch-predictor/blob/master/report/main.pdf}{Branch Predictor}}
  % {Best Predictor}
  % {April'2018}
  % {\emph{\texttt{\href{https://github.com/yashsriv/branch-predictor/blob/master/report/main.pdf}{github://yashsriv/branch-predictor}}}}
  % {
  %   \begin{cvitems}
  %   \item Designed a branch predictor for an intra-class branch prediction
  %     championship based on the CBP-1 framework in a team of 2.
  %   \item Created a modified GEHL predictor with an additional loop predictor.
  %   \item Was adjudged the \textbf{best predictor} amongst all submitted.
  %   \end{cvitems}
  % }

  % \smallcventry
  % {Course Project}
  % {\href{https://github.com/yashsriv/haskell-connect-4}{Connect 4 AI in haskell}}
  % {Functional Programming}
  % {Jan'2018-April'2018}
  % {\emph{\texttt{\href{https://github.com/yashsriv/haskell-connect-4}{github://yashsriv/haskell-connect-4}}}}
  % {
  %   \begin{cvitems}
  %   \item A GUI based connect 4 AI in haskell.
  %   \item Had support for various difficulties and the AI was abstracted out in
  %     order to be able to support any complete knowledge two player game.
  %   \end{cvitems}
  % }

  % \smallcventry
  % {24 Hour Hackathon}
  % {Code.Fun.Do}
  % {Microsoft India}
  % {Sept'2015}
  % {Best 5 ideas}
  % {
  %   \begin{cvitems}
  %   \item Developed an App to help connect teachers and learners based on their
  %     preference of subjects.
  %   \item Used cross-platform \textbf{Universal App Platform} for Windows 10
  %     and a server written in C\#.
  %   \item Was selected as one of the \textbf{best five ideas}.
  %   \end{cvitems}
  % }

  % \smallcventry
  % {Self Project}
  % {\href{https://github.com/yashsriv/go-nachos}{go-nachos}}
  % {Ported nachOS to golang}
  % {Dec'2017}
  % {\emph{\texttt{\href{https://github.com/yashsriv/go-nachos}{github://yashsriv/go-nachos}}}}
  % {}
\cvsection{Mini/Self Projects}


\begin{itemize}
%\item \textbf{\href{https://github.com/RohitRanjangit/PolySAT}{PolySAT}}, Implemented a SAT solver for propositional logic using the DPLL Algorithm in C++
\item \textbf{\href{https://github.com/RohitRanjangit/FlappyBirdAI}{FlappyBirdAI}}, Created a FlappyBird AI model using NEAT Algorithm
%\item \textbf{\href{https://github.com/RohitRanjangit/chessPY}{ChessPy}}, Created a very simple chess engine using Python
\item \textbf{\href{https://github.com/RohitRanjangit/StudentData}{StudentDATA}}, Developed a system to handle student database in C++ using SQL Database
\item \textbf{\href{https://github.com/RohitRanjangit/Decoder}{Decoder}}, Developed a decoder in Haskell to decipher monoalphabetic substitution ciphers
\end{itemize}


\cvsection{Miscellaneous}

\begin{itemize}
  \item Mentored Freshers in ACA project \textbf{FlappyBirdAI using NEAT} organized by Computer Science department 
  \item \textbf{Senior Marketing Executive} in annual sports festival Udghosh’19, IIT Kanpur
  \item \textbf{Senior Web Executive} in annual sports festival Udghosh’19, IIT Kanpur
  \item Received \textbf{First} Prize, National Level Chess Championship held at Nagaon, Assam
\end{itemize}
% \cvsection{Interests}

{\fontsize{11pt}{1em}\bodyfontlight\upshape\color{text}
  \begin{itemize}
  \item Open Source
  \item Capture The Flag Contests
  \item Web Development
  \item Image Processing
  \item Artificial Intelligence
  \item Robotics
  \end{itemize}
}

%%% Local Variables:
%%% mode: latex
%%% TeX-engine: xetex
%%% TeX-master: "../cv"
%%% End:


\end{document}

%%% Local Variables:
%%% mode: latex
%%% TeX-engine: xetex
%%% End: